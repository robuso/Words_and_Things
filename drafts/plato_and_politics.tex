% !TEX TS-program = xelatex
% !TEX encoding = UTF-8 Unicode

\documentclass[11pt,a4paper,english]{article} % document type and language

\usepackage{fontspec}

\setmainfont{Linux Libertine O}

\begin{document}
    

\section{Plato and Politics}

I remember in 2016, when I was on my way out from teaching in grad school, observing that teachers of Plato's \emph{Republic} would be helped along greatly by having a living example of the tyrant with which to teach. Trump \emph{is} the appetitive soul, and we have a front-row seat to Plato's political and moral philosophy, whether we'd like to or not. In the spirit of making lemons into lemonade, and see half-full glasses, I think it's appropriate to face this reality, and to understand the tripartite soul in its modern context.

What people often remember about the \emph{Republic} is Plato's `Allegory of the Cave', and often in a context somewhat removed from Plato's original intention. The typical framing of the allegory is as a model of our limited perceptual, or even conceptual capacities. Here is how Plato (speaking through the character of Socrates) describes the cave to Glaucon:
\begin{quote}
    Imagine human beings living in an underground, cavelike dwelling, with an entrance a long way up that is open to the light and as wide as the cave itself. They have been there since childhood, with their necks and legs fettered, so that they are fixed in the same place, able to see only in from of them, because their fetter prevent them from turning their heads around. Light is provided by a fire burning far above and behind them. Between the prisoners and the fire, there is an elevated road stretching. Imagine that along this road a low wall has been built---like the screen in front of people that is provided by puppeteers, and above which they show their puppets. (Book 7, 514a--b)
\end{quote}
As the story goes, one of these prisoners breaks free from her fetters, and journeys up and out of the cave---discovering the real world of which her previous life was (literally) a mere shadow.\footnote{In fact, it's worse than that, since it's a \emph{shadow} of a \emph{replica} of the real world. But that's another story\dots}

This has been, understandably and rightfully, an iconic and influential vingette that summarizes \emph{some} of Plato's arguments in the \emph{Republic}. However, 

\end{document}

% pandoc a_feast.tex -f latex -t html -s -o a_feast.html